\documentclass{NIHGrant}
\begin{document}
\part*{Training in the Responsible Conduct of Research}
In addition to its emphasis on the humanities, PSCOM invests a significant amount of resources and training time in ensuring that students and faculty who conduct research within the University are well-versed in ethical conduct even in sensitive and difficult materials.

\section*{Biomedical Research Ethics (BMS 591)}
All graduate students at Penn State take an ethics course within the first two years of graduate school to set the standard for their research and scholarship. BMS 591 is an in-person course offered at Hershey that employs weekly quizzes, guest lecturers, and team-based learning (TBL) to instill a rigorous understanding of ethical principles among students. The course meets weekly on Fridays and covers a new topic each week. Students are provided with assigned prereading from the textbook the \textit{Introduction to the Responsible Conduct of Research} by Nicholas Steneck. Additionally, the course directors provide students with additional summative powerpoint presentations, news articles, and other media that they are expected to have studied ahead of taking an individual closed-book quiz at the start of the class each week. A passing grade in the class is required for progression through graduate programs at Penn State. I completed this course in the Fall semester of 2023 and scored an A.

\section*{Collaborative Institutional Training Initiative (CITI):}
PSCOM requires students and faculty to study online modules that cover standard operating procedures for sensitive materials such as research involving bloodborne pathogens and human subjects research. Each module contains a written and pre-recorded video component that instructs researchers and clinicians on how to ethically and responsibly navigate research studies and potentially hazardous situations in the laboratory. To receive certification, participants must take a quiz and receive a score of at least 80 percent. Through the CITI training portal I have completed and received certification for ``Protection of Human Research Subjects'', ``Responsible Conduct of Research'', ``OSHA Bloodborne Pathogens'', and more. Throughout the duration of this award I will continue to maintain these certifications and complete additional ethics and safety training as new experiments are founded.

\section*{Institutional Review Board (IRB) Study:}
I have applied and received approval for an IRB study that will cover imaging experiments using human prostate cancer samples. This IRB was compiled under the direction of Dr. Joshua Warrick (see reference letter and biosketch), and I received hands-on training in the responsible conduct of research with a focus on the management of protected health information (PHI). Our approved study is evidence of our productive collaboration and sets a precedent for a thorough understanding of the responsible conduct of research throughout the execution of the proposed work.

\section*{PSCOM MSTP Program Annual Responsible Conduct in Research Seminar:}
All students in the MD/PhD program at PSCOM annually attend a seminar where they review responsible conduct of research in a didactic format. Each session is led by the Ethics Consultation Service at PSCOM and is required for MD/PhD students.

\section*{Silverman Lab Meetings:}
The Silverman lab hosts regular discussions on responsible data analysis and ethical use of statistical models. Our weekly lab meeting regularly involves discussion of literature articles in the field of sequence count data, conformal prediction, and causal inference. Dr. Silverman and Dr. Michelle Nixon regularly provide instruction at the whiteboard to review concepts in statistics and walk the lab through discussion of logical and rigorous interpretation of data.

\section*{Cheng Lab Meetings:}
The Cheng lab also covers topics in the responsible conduct of research during its meetings, especially in the context of accurate representation of data and the biological context of images. Dr. Cheng routinely reviews medically relevant concepts such as the fundamentals of histopathology with grad students and emphasizes rigor and reproducibility in image reconstruction and interpretation.
\bibliographystyle{IEEEtran}
\bibliography{refs.bib} %Entries are in the refs.bib file
\end{document}
