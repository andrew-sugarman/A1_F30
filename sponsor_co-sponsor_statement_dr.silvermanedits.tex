\documentclass{NIHGrant}

\usepackage{mathtools}
\usepackage{subcaption}


\newtheorem{theorem}{Theorem}[section] \newtheorem{definition}{Definition}[section]
\newtheorem{corollary}{Corollary}[theorem] \newtheorem{lemma}[theorem]{Lemma}

%% Macros and Aliases
\let\inf\rest \DeclareMathOperator*{\inf}{Inf} % Infimum, note: * undersets limits
\newcommand{\E}{\mathbb{E}} % Nice Infimum
\newcommand{\R}{\mathtt{R}} % Real Space
\renewcommand{\S}{\mathtt{S}} % Simplex
\newcommand{\Z}{\mathtt{Z}} % Integers
\DeclareMathOperator*{\mean}{mean}
\DeclareMathOperator*{\median}{median}


\usepackage{microtype}

\usepackage{floatrow}
\DeclareFloatFont{tiny}{\tiny}% "scriptsize" is defined by floatrow, "tiny" not
\floatsetup[table]{font=tiny}

\begin{document}

\thispagestyle{empty}

\part*{Sponsor and Co-Sponsor Statements}
\vspace{10pt}

\section*{Justin D. Silverman, M.D., Ph.D. }

\subsection*{Research Support Available}

\begin{center}
\renewcommand{\arraystretch}{1.3}
  \fontsize{9pt}{9pt}\selectfont
  \begin{tabular}{| p{0.5in} | c | p{1.5in} | p{1in} | p{0.5in} | p{0.5in} | p{0.7in} |}
    \hline 
    \textbf{Source} & \textbf{ID} & \textbf{Title} & \textbf{PD/PI} & \textbf{Start} & \textbf{End} & \textbf{Amount} \\
    \hline
    NIH & 1R01GM148972-01 & Addressing Measurement Limitations for Sequence Count Data &  Silverman, JD & 9/2022 & 8/2025 & \$599,945 \\
    \hline
    PNNL & 1R01GM149650-01 & A Novel high resolution MS platform for high-throughput screening of G protein- coupled receptors & Jacobs, J \& Rajagopal S & 6/2023 & 3/2027 & \$86,000 \\
    \hline 
    Duke Univ  & 2R01DK116187-06A1 & Dietary plant diversity and the human gut microbiome & David, LA & 5/2024 & 5/2028 & \$168,134 \\
    \hline
  \end{tabular}
\end{center}

Beyond the funding listed above and other proposal that are currently submitted
or planned, Dr. Silverman has over \$300,000 in startup funds from The College
of Information Science and Technology (IST) at the Pennsylvania State University
which can be used to support Mr. Sugarman's research and training. Dr. Silverman
also holds the Diana and Raymond Tronzo Medical Informatics Endowment through
the College of IST which can be used to support student researchers.

\subsection*{Sponsor's Previous Fellows/Trainees}

I have trained and mentored two doctoral students (both graduated with Ph.D. degrees) and three masters students (all graduated with M.S. degrees) and am currently training or mentoring 5 doctoral students. Of the two doctoral students that have graduated from my lab, one currently holds a senior position in industry and the other is pursuing postdoctoral education. 

\begin{center}
\renewcommand{\arraystretch}{1.3}
  \fontsize{9pt}{9pt}\selectfont
  \begin{tabular}{| l | p{2in} | p{3in} |}
    \hline
    \textbf{Name} & \textbf{Position in Silverman Lab} & \textbf{Current Position and Institution} \\
    \hline
    Kimberly Roche, Ph.D. & Predoctoral & Senior Translational Scientist at Tempus Labs \\
    \hline
    Emily Van Syoc, Ph.D. & Predoctoral &  Postdoctoral Research, PSU \\
    \hline
    Farhani Momotaz, M.S. & Masters Student & Research Associate, PSU \\
    \hline 
    Zhao Ma, M.S. & Masters Student & PhD Student, University of Texas, Dallas \\
    \hline
    Manan Saxena & Masters Student & Research Assistant, PSU\\
    \hline
  \end{tabular}

\end{center}

\section*{Keith C. Cheng, M.D., Ph.D. }

\subsection*{Research Support Available}

My lab also has substantial research support that will provide Andrew with a surplus of the resources needed to carry out this proposal. In addition to our track record of NIH funding, our group maintains active collaboration with synchrotron beamlines such as the Lawrence Berkeley National Laboratory where Andrew has collected preliminary data and will continue to carry out experiments to support the proposed work.

\begin{center}
\renewcommand{\arraystretch}{1.3}
  \fontsize{9pt}{9pt}\selectfont
  \begin{tabular}{| p{0.5in} | c | p{1.5in} | p{1in} | p{0.5in} | p{0.5in} | p{0.7in} |}
    \hline 
    \textbf{Source} & \textbf{ID} & \textbf{Title} & \textbf{PD/PI} & \textbf{Start} & \textbf{End} & \textbf{Amount} \\
    \hline
    NIH & 1R24OD18559 & Groundwork for a Synchrotron MicroCT Imaging Resource for Biology &  Cheng, KC & 8/2015  & 7/2024 (NCE)&\$2,680,046 \\
    \hline
    NIH & R24OD18559 & Renovation Supplement to Groundwork for a Synchrotron MicroCT Imaging Resource for Biology &  Cheng, KC & 8/2022  & 7/2024 & \$231,489 \\
    \hline
    DOE/ LBNL & ALS-11922 & X-ray histotomography applications of new wide-field, submicron resolution lens and camera systems & Cheng, KC & 7/2022 & 6/2025 & Synchrotron Imaging Time \\
    \hline 
    NIH & 1R24OD035407-01A1 & Building a Wide-field, High-resolution Histotomography Resource for Biology & Cheng, KC & 6/2024 & 5/2028 & \$3,862,968 \\
    \hline
  \end{tabular}
\end{center}

\subsection*{Co-Sponsor's Previous Fellows/Trainees}

I train and mentor doctoral (including MD, Veterinary, PhD, and post-graduate) students in multiple fields, including Genetics, Biomedical Sciences, Pathology, Bioinformatics and Genomics, and Anatomy, including several MD/PhD students. As an active participant in intercollege and interdisciplinary programs including the Huck Institute for the Life Sciences and Institute for Computational and Data Sciences involving undergraduate and graduate students and faculty on multiple Penn State's campuses, and have served on graduate committees for students in Information Sciences and Technology and Computer Sciences, I interact with students at multiple levels of training and fields.  I am currently training or mentoring 3 doctoral students, a postdoctoral student, a post-baccalaureate student, and 1 Assistant Research Professor. My mentees include the following:

\begin{center}
\renewcommand{\arraystretch}{1.3}
  \fontsize{9pt}{9pt}\selectfont
  \begin{tabular}{| l | p{3in} | p{5in} |}
    \hline
    \textbf{Name} & \textbf{Current Position and Institution} \\
    \hline
    Rebecca Lamason, Ph.D. (postbaccalaureate) & Associate Professor of Biology, MIT \\
    \hline
    Darin Clark, Ph.D. (postbaccalaureate) & Assistant Professor of Radiology, Duke Univ \\
   \hline
    Amogh Adishesha, Ph.D. (IST)& Applied Scientist, Captions (captions.ai) \\
    \hline
    William Zinnanti, M.D., Ph.D. (Biomed Sci) & Private Practice for Child and Adult Neurology\\
    \hline
    Brian Canada, Ph.D. (Bioinformatics/Genomics) & Chair, Department of Computer Science, Univ of South Carolina \\
    \hline
    Yifu Ding, M.D. Ph.D (Biomed Sci) & Resident Physician in Radiation Oncology, Emory University \\
    \hline 
    Spencer Katz, M.D. Ph.D (Biomed Sci) & Resident Physician in Pediatric Medical Genetics, Cincinnati Children's Hospital \\
    \hline
    Maksim Yakovlev, Ph.D (Biomed Sci) & Postdoctoral Researcher, Argonne National Laboratory \\
    \hline
  \end{tabular}

\end{center}

\subsection*{Sponsor Statement}
\subsection*{Training Plan, Environment, Research Facilities}

\subsubsection*{Training Plan: }
The goal of this F30 application is to formally train Andrew in basic and translational sciences. He will continue to receive rigorous and comprehensive training from a variety of sources including coursework, one-on-one mentorship, discussion with other graduate students, and group meetings with Dr. Silverman and Dr. Cheng and Synchroton and University Imaging Groups.

Andrew has completed and excelled in all formal coursework requirements for the Bioinformatics and Genomics PhD program and has passed his comprehensive exam. These courses have included a variety of core topics in computer science and biostatistics. Andrew has learned the methodological skills needed to complete the experimental portions of this proposal: he has developed expertise in sample-preparation micro-CT physics, and image processing. Under the direct supervision of Dr. Silverman, Andrew continues a rigorous self-study program in statistics and machine learning. To date, Andrew has finished a self-study (along with problem-sets) from the core textbook by Capella and Berger entitled \textit{Statistical Inference}. He has also learned foundational concepts in Bayesian statistics from Peter Hoff's \textit{A First Course in Bayesian Statistics}. He has completed a study of topological data analysis centered around the textbook by Edelsbrunner and Harer titled ``Computational Topology: An Introduction''. Beyond these topics, I have personally taught Andrew the foundations of functional data analysis and matrix algebra required for successful completion of Aim 3. Supplementing this, I will oversee Andrews progress as he takes graduate courses in topological data analysis and machine learning this fall.

\subsubsection*{Clinical Training Plan: }
Both Dr. Cheng and I have completed clinical training and understand the importance of continuing to cultivate knowledge and patient-centered clinical skills during graduate studies. Investing in clinical training during Andrew's graduate years will prepare him to transition back to the third year of medical school and his long-term goal of becoming an independent physician-scientist. This will enhance the value of his graduate studies, given that Andrew draws substantial motivation from what he has observed in the clinic and has initiated this proposal with the hope that it will be able to contribute to problems that affect cancer patients.

Andrew participates in the Clinical Exposure Program (CEP) since he has completed his comprehensive exam. CEP is a tool for MD/PhD students to prepare for the clinical portions of medical training while they complete their PhDs. He has chosen to work with Dr. Lilia Reyes in the Pediatric Emergency Department to focus on becoming a well-rounded medical student and ultimately a versatile physician scientist. To supplement this work, he will also continue to shadow Dr. Raymond Hohl, an attending in hematology/oncology at Hershey and the director of the Penn State Cancer Institute, and a member of Andrew's thesis committee.

\subsubsection*{Professional Development Plan: }
Andrew will continue to have career development opportunities through presenting at multiple conferences, seminars, and workshops and participating in unique research-focused lab trips. He has already presented his work at the MD/PhD National Student Conference and the Bioinformatics and Genomics program retreat. Presentation skills are essential components of a successful career in science, and Andrew will have ample opportunities to develop strong scientific communication abilities in both of our labs.

We will be sure that Andrew is also able to attend the MD/PhD retreat twice a year, where he will observe and interact with keynote speakers and colleagues that will also contribute mentorship to his development as a physician-scientist. Wherever possible, Andrew will also engage in other presentations and seminars on campus such as the Graduate Student Research Forum, the Bioinformatics and Genomics Student Seminar, and the Experimental Pathology Colloquium.

Dr. Silverman meets with his advisees weekly in one-on-one meetings for at least
1 hour. In addition to these one-on-one meetings, the Silverman lab meets
weekly. These weekly lab meetings are devoted to the development and application
of statistical methods for complex biomedical data. Lab meetings alternative
every other week between journal clubs and student and faculty presentations.
Meetings are two hours, giving enough time for both presentation and discussion
among the group. Presenters rotate with each student presenting once per
semester. Lab meetings and one-on-one meetings are both held more frequently
during key periods such as during manuscript preparation, preparation for
committee meetings, or preparation for conference presentations. We have
established Andrews's thesis committee which meets annually to discuss his
progress. Beyond these annual meetings Andrew meets regularly with each
committee member to discuss aspects of his research that intersect with their
own expertise.

Andrew has access and regularly interacts with a wide range of extramurally
funded researchers with expertise directly applicable to the proposed work and
to Andrew's larger career goals. I am an assistant professor in the College of
Information Science and Technology (IST), the Department of Statistics, and the
Department of Medicine. I have formal training in both statistics (PhD) and
medicine (MD) which gives me a keen appreciation of both the translational
context of key biomedical questions as well as the statistical challenges
associated with answering those questions. I am currently the PI on an NIH R01
award which focuses on developing key theory and tools for scale reliant
inference that are complementary yet non-overlapping with the proposed work
(1R01GM148972-01). Other extramurally funded faculty that Andrew regularly
interacts with include Dr. Francesca Chiaromonte (Statistics, Focus on
Bioinformatics and Genomics), Dr. Matthew Reimher (Statistics, Focus on
Functional Methods for Biostatistics), and Dr. Vasant Hanovar (IST, Focus on
Statistical Inference and Machine Learning for Biomedical Data).

Outside of the lab and his thesis committee, there are numerous venues on campus
that regularly feature current work from extra and intra-mural researchers.
These include the Bioinformatics and Genomics Colloquium Series, The Statistics
Seminar Series, and the Microbiome Center Seminar Series. 


\subsubsection*{Research Facilities: }

Andrew will have access to the facilities and resources necessary to undertake
and complete the proposed work. Andrew has dedicated office space in my lab
located in the Westgate building at the Pennsylvania State University. This
building is modern (built in 2004 and recently renovated), centrally located on
campus, and within a 5 minute walk from the Statistics department and the Huck
Life Sciences Institute. Andrew's office space is just down the hall from my
own. Andrew has access to numerous high performance computing clusters both
through the College of IST as well as through the Institute for Computational
and Data Science (ICDS). As a co-hire with the ICDS, students in the Silverman
lab have priority access to the Roar computing cluster which contains over
36,500 computing cores and 25 PB of storage. To supplement these capabilities
the Roar cluster also has a help-desk service (i-Ask) which assists users will
all aspects of cluster computing from debugging job execution to systems
administration and database maintenance. Beyond computing resources, the Penn
State University Libraries rank among the top 10 North American research
libraries based on the Association of Research Libraries Library Investment
Index Rankings. The library system consists of 36 libraries at 24 locations
throughout the Commonwealth of Pennsylvania. The University Libraries house a
collection of nearly 6 million items, with annual additions of roughly 100,000
volumes. The libraries have access to 579 online databases and other e-resources
and subscribe to nearly 118,000 online, full-text journals.


\subsection*{Number of Fellows/Trainees to be Supervised During the Fellowship}

Four, in addition to Andrew: 

\begin{itemize}
\item Tinghua Chen, Graduate Student (Informatics)
\item Kyle McGovern, Graduate Student (Bioinformatics and Genomics)
\item Won Gu, Graduate Student (Statistics)
\item Maxwell Konnaris, Graduate Student (Bioinformatics and Genomics)
\end{itemize}

\subsection*{Applicant's Qualifications and Potential for a Research Career}

While I have only formally advised Andrew for the past two years (since he
started in the PhD portion of his training), it is important to note that I have
been working with Andrew for almost four years -- since he initially rotated in
my lab in the fall of 2020. In that time, I have watched Andrew combine his
intellect, creativity, and grit to great effect. For example, Andrew learned the
foundations of probability theory, Frequentist statistics, and topological data
analysis in just over two months after he decided he wanted to learn how to
rigorously model shape in 3D images. This is a remarkable feat and, in my
experience, tantamount to crushing a wall that other students would simply walk
away from. Andrew displays incredible intellectual curiosity has the grit and
intellect to follow that curiosity. Combined with his long-term interests in
cancer, I have no doubt that Andrew will ultimately lead a independent research
program that will improve human health.

\subsection*{Co-sponsor Statement: }
It is important to note that the development of a 3D computational phenomics for cancer is at its start, highly technical, and involves a degree of interdisciplinary collaboration that is challenging and non-traditional. The first reason this is important is that the increasing complexity of science means that trainees like Andrew who are trained to handle this complexity, will likely be enabled to make unique and important contributions that strictly focused scientists cannot.

I particularly enjoy the pursuit of science driven by first principles and interdisciplinary collaboration. In my experience, only the rare graduate student has sufficient intellect, breadth of knowledge, and commitment to see through this approach. As I detected even during interview, Andrew is one of such students. In this case, the project needed breadth of understanding to connect patient care with tissue structure and computing. He had computing chops well-illustrated by his involvement in a structural RNA-drug interaction project published in Nature Communications. He understood that he would need to understand enough pathology and genetics put his work into broad biological context. He has successfully come to and biophysics to understand the basis of sample preparation, learn enough micro-CT physics using both cone-beam based local and parallel-beam based synchrotron x-ray sources, image processing, machine learning, and interacting in an multidisciplinary environment - a strength of the Cheng lab.

Andrew caught on quickly in terms of all these areas, being aware of, but never shying away from the complexity.  He showed leadership skills in first learning how synchrotron micro-CT imaging trips work, and then being a primary manager of the most recent experimental trip to the LBNL. He led discussion of phase-based, unstained sample micro-CT based on readings from the literature as personally guided by Dr. Cheng. The striking preliminary data inspired and are foundational to this proposal; its extensions across tissue and sample types will contribute to multiple publications. Please also note that not only has Andrew already participated productively in other publications in the lab, he is also contributing to the projects of others in the lab in a way that will be mutually beneficial and result in multiple collaborative publications. I have mentored numerous PhD and MD/PhD students, and have significant experience and understanding of the various types of qualities that can drive success in interdisciplinary science. Andrew possesses a wonderful set of qualities, giving me every faith that he will become an excellent physician-scientist and make impactful contributions to cancer research.

\end{document}


backup table

\begin{center}
  \begin{tabular}{| p{0.5in} | c | p{1.5in} | p{1in} | c | c | c |}
    \hline 
    Source & ID & Title & PD/PI & Start & End & Amount \\
    \hline
    NIH & 1R24OD18559 & Groundwork for a Synchrotron MicroCT Imaging Resource for Biology (SMIRB) &  Cheng, KC & 08/15  & 7/23 &\ \\
    \hline
    NIH & R24OD18559 & Supplement to Groundwork for a Synchrotron MicroCT Imaging Resource for Biology (SMIRB) &  Cheng, KC & 08/19  & 7/24 &\ \\
    \hline
    DOE/ LBNL & ALS-11922 & X-ray histotomography applications of new wide-field, submicron resolution lens and camera systems & Cheng, KC & 7/2022 & 6/2025 & \ \\
    \hline 
    NIH & 1R24OD035407-01A1 & Building a Wide-field, High-resolution Histotomography Resource for Biology & Cheng, KC & 5/2024 & 5/2028 & \$3,862,968
    \\
    \hline
  \end{tabular}
\end{center}
