\documentclass{nihgrant}
\begin{document}
\part*{Introduction to Resubmission}

I appreciate the reviewers' insightful feedback regarding my F30, which I believe has improved the quality of this resubmission. Below I provide a high-level summary of key comments from the reviewers and how I have addressed them in this resubmission.

% is it appropriate to mention work from prior labs
One reviewer expressed concern regarding my lack of first-author publications. In the revised submission, I provide additional evidence for my productivity throughout my PhD training. I am currently preparing two first-author manuscripts centered around the techniques I present in this proposal. The first manuscript titled \textit{Whole-organism neural connections in Octopus by centimeter-field, sub-micrometer X-ray histotomography} will be submitted to \textit{Current Biology} by the end of August. This article illustrates how our novel wide-field micro-CT platform will contribute new insights in biology. The second manuscript titled \textit{Wide-field, propagation-based phase-contrast micro-CT reveals 3D glandular architecture in prostate cancer} reports how our adaptation of propagation-based phase-contrast imaging permits the whole-volume interrogation of prostate cancer biopsies without the need for complex staining procedures or tissue sectioning. We plan to submit this to \textit{Modern Pathology} by the end of October. Beyond these first-author projects, I have also earned co-authorsip on two additional publications: in \textit{eLife} and another in currently in review.

Two reviewers expressed concern that the roles of the co-sponsors were not clearly defined. In the revised submission, I have clarified their roles. Dr. Silverman is the primary sponsor of this project and will oversee the majority of the proposed work. Supervised by Dr. Cheng, I have already developed the phase-contrast imaging technique and imaged the majority of the proposed samples. The bulk of the remaining work is computational and, as a result, Dr. Silverman is the primary sponsor of this project. More specifically, Dr. Cheng will supervise the remainder of Aim 1 and Dr. Silverman will supervise Aim 3. Regarding Aim 2, Dr. Cheng and Dr. Silverman will supervise study design, Dr. Cheng will assist in participant recruitment, and Dr. Silverman will supervise analysis of study data. Dr. Silverman will also supervise my continued self-study in Topological Data Analysis (TDA) and other aspects of statistics and machine learning.

One reviewer commented that the proposed work seemed to be a departure from my training background. In the revised submission, I provide greater detail regarding my background and research training to date, indicating how my past, current, and future experience align with the proposed work. In brief, my undergraduate training focused on computational methods within biochemistry. As a part of my PhD, I have taken couses in computational methods, algorithms, and statistics. I am also two years into an intensive three year self-study program in statistics and machine learning supervised by Dr. Silverman. This self-study has included discussions, assignments, and readings. Finally, in the coming year I will take graduate courses in Machine Learning and TDA.

% should it be repeated that'' `we have sufficient preliminary data to begin work on each aim simultaneously right now''?``''
One reviewer expressed concern that the aims were interdependent. In the revised submission, I highlight that I have collected enough preliminary data to make the aims functionally independent yet still synergistic. While the planned additional data will bolster Aims 1-3, success in those aims is not dependent on that additional data. I include additional preliminary data in support of aim 1 to indicate its feasibility. We highlight the versatility of TDA and its value as the field of 3D histopathology expands.
% alternative strategies for serial section?

One reviewer questioned whether this project would be appropriate given my long-term focus on oncology rather than clinical pathology. The proposed work details novel methods and study of morphological characterization in cancer. In the revised submission, I have clarified how the proposed research and training fit my long-term goals to lead an independent research program in oncology as a research scientist. While the proposed research could be well suited for a career in pathology, the technology and foundational insights gained will provide foundational insights into cancer biology and are highly relevant to my long-term goal of advancing the treatment of cancer patients. The revised clinical training plan emphasizes oncology (through coursework and elective rotations) alongside hands-on pathology experience and will position me to train as a leader in either field.  % Echoing the above comment, the sponsor has been chosen based on where the majority of the oversight will be required. The imaging is largely complete at this point, and I foresee the area that will need the most direct supervision will be in the analysis and design of experiments. In consultation with Dr. Cheng and Dr. Silverman, roles were designated with this in mind.

% We acknowledge the lack of ... and the sectoin of the training plan has been edited
  % this project focuses on the fundamental morphologic characterization of cancer ... link this with long term career goals in hematology oncology
One reviewer requested additional biosketches from the referees for this proposal. In the updated application, updated biosketches for Dr. Warrick, Dr. La Riviere, and Dr. Tolbert are included. We are very thankful for their contributions and ongoing support for this project.

\vspace{2.5mm}

\noindent Thank you for your valuable feedback and consideration of my revised application.

\noindent Sincerely,

\noindent Andrew Sugarman

\end{document}

OUTTAKES:
 In the revised submission, I reiterate that my long-term goal is to lead an independent research program in oncology as a physician scientist.
