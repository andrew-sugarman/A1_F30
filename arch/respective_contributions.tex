\documentclass{NIHGrant}
\begin{document}

\part*{Respective Contributions}
The goals and expected outcomes of the proposed research build on studies published in the Cheng Lab, specifically the work by prior MD/PhD student Yifu Ding (Ding et. al 2019). However, mico-CT in general has only been applied to soft tissue cancer research relatively recently. In developing this proposal, we determined a gap in the field that our lab was uniquely positioned to contribute to. My long term goals are centered around contributing to cancer research, and I identified micro-CT as a technology with increasing capability to improve our ability to measure and even diagnose malignancy. \emph{This proposal builds on current literature in micro-CT and combines the technology with novel statistical methods to better understand prostate cancer and other malignancies.} This work does so while \emph{prioritizing my training in statistics and programming.} In writing this proposal, I read Dr. Silverman's R01 (1R01GM148972-01) and studied the structure of his writing and the strategy with which he supported his aims. I have been discussing the elements of this proposal since July of 2023. I defended my comprehensive exam in August and incorporated feedback from my committee to refine the aims of my research. Through joint discussions with Dr. Warrick (see reference letter) and Dr. Silverman in the wake of my exam, we decided to pivot to make prostate cancer a focus of this proposal. I worked diligently with Dr. Warrick (see letter of reference) to write and submit an IRB proposal ahead of a trip I led to the LBNL in early October of this year. Our IRB application was approved and Dr. Warrick was able to identify samples that I went on to prepare and image at beamline 8.3.2 (data shown in figure 2 of this proposal). I collaborated with Dr. Patrick La Riviere (see letter of reference).

The preliminary data collected under the direction of Dr. Warrick supports the feasibility of this proposal and I will continue to build on this work as I begin with aim 1. Dr. Warrick, Dr. Gunther, and Dr. Hu will contribute samples for aim 1. Under the direction and guidance of Dr. Cheng and Dr. La Riviere, I will conduct micro-CT imaging acquisition and reconstruction in aim 1, along with data acquisition, visualization, and processing for aim 2. Dr. Silverman and Dr. Warrick will mentor and assist me through the study design and statistical analysis of aim 2, and Dr. Silverman will directly oversee algorithm development and programming in aim 3. I will review results and prepare manuscripts in close discussion with Dr. Silverman, Dr. Cheng, Dr. Warrick, and Dr. La Riviere. I will review data and critique with additional support faculty members such as Dr. Jiafen Hu and Dr. Edward Gunther (see letters of support) to improve my knowledge, writing ability, and navigate the peer review process. I am thankful to be supported by a multi-disciplinary research team that is led by the close guidance of Dr. Silverman, who has set a great example from the start of how I aim to run my own lab when I am an independent investigator.
\bibliographystyle{IEEEtran}
\bibliography{refs.bib} %Entries are in the refs.bib file
\end{document}

%%% ask Caryn again about where to include IRB study information if at All
%%% mention prelim data in Dr. warrick letter
%%% fix RO1 number?
