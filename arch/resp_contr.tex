\documentclass{NIHGrant}
\begin{document}

\part*{Respective Contributions}
% elaborate on post-comprehensive discussions  with Dr. Warrick?
% is it reasonable to say this proposed work is modeled after silverman R01?
% need a single sentence overview
\begin{itemize}
        * When updating the research strategy make it explicitly clear at the end (given our preliminary data we are ready to begin work on EACh of these aims. They are independent but functionally synergistic)

1. Highlight that I catalyzed this collaboration dirven by your interests and supported by your advisors
2. The three of us regularly meet (Dr. Silverman + keith regularly meet )
3. this is a multidisciplinary project and I have already acquired skills across disciplines (synch imaging, data science work elife)
         4. what have been my contributions - how have the advisors supported that, and how they will support
         5. dr. cheng imaging resources technique facilities
         6. dr. warrick framed initial research question and provided samples
         7. dr silverman - there are goals I have set that led me to seek out Dr. silverman
         8. the ultimate project is an interdisciplinary collaboration highlighting the expertise of my mentorship team, driven by my own interests in technically demanding cancer research. Dr. so and so's contributions are

 \end{itemize}

Modeled after foundational questions intersecting Dr. Warrick's clinical knowledge
The proposed research strategy is modeled after the questions investigated by the Silverman Lab. The group investigates the measurement of complex biomedical data (tools for sequence count data analysis, applied Bayesian statistics, partially identified models), working to develop statistical methods for reliable inference in the case of high-dimensional, unconventional data. Justin Silverman MD, PhD is the primary sponsor of this project as we set out to improve the understanding of a fundamental problem in cancer biology and treatment by leveraging unique yet rigorous methods development with broad implications.
% the silverman lab investigates methods of measuring complex biomedical data, and subsequently develops creative solutions to enhance their rigor

The goals and expected outcomes of the proposed research also build on studies published in the Cheng Lab, specifically the work by graduated students Yifu Ding MD, PhD (Ding et. al 2019), Spencer Katz MD, PhD (Katz et. al 2021), and Maksim Yakovlev PhD (Yakovlev et. al 2023, 2024). However, mico-CT in general has only been applied to soft tissue cancer research relatively recently. In developing this proposal, I conducted a literature review and observed two critical gaps in knowledge that our team was uniquely positioned to contribute to: identifying repeatable phase-contrast imaging parameters capable of generating 3D images of whole-biopsies at diagnostic resolution and a need for rigorous, interpretable means of quantifying the phenotypes within these datasets. These two research questions are directly relevant to my long term goal of contributing to cancer research, and I identified micro-CT as a technology with increasing capability to improve our ability to measure and potentially diagnose malignancy. \emph{This proposal builds on current literature in micro-CT and combines the technology with novel statistical methods to advance our understanding prostate cancer and other malignancies.} This work does so while \emph{prioritizing my training in statistics and programming.}

In support of Aim 1, Dr. Cheng and Dr. La Riviere (see letter of reference, biosketch) have provided me with training in advanced micro-CT imaging techniques, supporting my attendance of 5 synchrotron trips. \emph{I have served as experimental lead for the most recent 2 trips to beamline 8.3.2 of the Advanced Light Source at Lawrence Berkeley National Laboratory (LBNL), during which I performed phase-contrast imaging experiments that resulted in the preliminary data presented in this application.} I performed image processing, 3D reconstruction, and data visualization of each sample. I continue to collaborate with Dr. La Riviere to interpret phase-contrast imaging parameters and explore refinement of reconstruction. \emph{I will work with Dr. La Riviere to learn from is expertise in x-ray physics and medical imaging, and he will be critical to insuring the rigor of phase-contrast CT experiments.} The preliminary data was compared to histology in collaboration with Dr. Warrick (see letter of reference and biosketch). \emph{Dr. Warrick will provide his expertise in the clinical diagnosis of cancer and interpretation tumor morphology to integrate aims 2 and 3 with the foremost problems facing patient care in prostate cancer}

Preliminary data supports the feasibility of this proposal and I will continue to build on this foundation with Aims 1 and 2. Dr. Warrick, Dr. Edward Gunther (see letter of support), and Dr. Jiafen Hu (see letter of support) will contribute additional samples as we investigate additional solid tumor phenotypes with PBCT. I will lead additional experiments at Lawrence Berkeley National Lab and Argonne National Lab to complete Aim 1, along with data acquisition, visualization, and processing for aim 2. Dr. Silverman and Dr. Warrick will mentor and assist me through the study design and statistical analysis of aim 2, and Dr. Silverman will directly oversee algorithm and method development in aim 3. I will review results and prepare manuscripts in close discussion with Dr. Silverman, Dr. Cheng, Dr. Warrick, and Dr. La Riviere. I am thankful to be supported by a multi-disciplinary research team that is led by the close guidance of Dr. Silverman, who has set a great example from the start of how I aim to run my own future lab as an independent investigator. The experience I will gain from the proposed work will be elementary to my goal of leading interdisciplinary collaborations at the cutting edge of imaging and cancer research.

\bibliographystyle{nihunsrt}
\bibliography{refs.bib} %Entries are in the refs.bib file
\end{document}



%%% ask Caryn again about where to include IRB study information if at All
%%% mention prelim data in Dr. warrick letter
%%% fix RO1 number?
