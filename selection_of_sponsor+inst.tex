\documentclass{NIHGrant}
\begin{document}
\part*{Selection of Sponsor and Institution}
% one page limit
I elected to pursue a career as a physician-scientist after successfully defending my undergraduate honors thesis and performing full time research as a member of the Tolbert lab. I applied broadly to programs with strong computational research opportunities. I interviewed at Penn State College of Medicine (PSCOM) and discovered that this institution not only offered a wealth of opportunities in computational research, but also provided a holistic, supportive, and thorough translational training environment. The clinical humanities curriculum at PSCOM is unmatched and interfaces with every aspect of our training - research and clinical initiatives are each grounded in problems that plague our patients.

I entered my MD/PhD journey with the goal of becoming a computational physician-scientist capable of leading a cancer research lab. \emph{I entered my laboratory rotations with this in mind, and sought a training environment where I would thoroughly learn advanced techniques in data science and my technical skills would be challenged.} This led me to pursue laboratory rotations in the Cheng and Silverman labs, ultimately forming the basis for my co-mentorship.

I first conducted a pre-matriculation rotation in the Cheng lab. Dr. Cheng (co-sponsor) is an outstanding research mentor with an extraordinary track record as a leader in his field. He has a long history not only of high-impact publications and NIH funding, but also of innovation in imaging and computation. The lab recently pioneered the use of novel micro-CT detectors, generating 3D datasets that were entirely unique in their resolution to field-of-view ratio. \uline{I saw an opportunity within this lab to conduct research on the 3D imaging of human cancer biopsies. The unique technology and collaborative research environment cemented my decision to pursue collaboration with Dr. Cheng.}

I completed my next laboratory rotation by spending a month working with Dr. Silverman in the summer between the first and second year of medical school. I decided to pursue a rotation in his lab because of my background and goals in computational work and the unique data analysis problems I had encountered during my first rotation, such as 3D segmentation within high-resolution images. Dr. Silverman immediately held me to a high standard while simultaneously investing time in walking me through detailed and difficult concepts in math and statistics such as Gaussian process regression, conformal prediction, and subjects in topological data analysis (TDA) such as the Vietoris-Rips filtration. \uline{To this day he does not hesitate to work through problems on the whiteboard with myself or any of his students, and it is this investment in trainees that has guided me as a grad student and motivated me to pursue a career leading a team of my own.} After this experience with Dr. Silverman, I was certain he was the ideal mentor for my graduate research with the optimal expertise and background to support me in navigating this collaboration.

Co-mentorship by Dr. Silverman (sponsor) and Dr. Cheng (co-sponsor) directly compliments the proposed research training plan and will uniquely position me to achieve my goals as a physician-scientist. Dr. Silverman will directly contribute to the challenging training in statistical methods and TDA required for aims 2 and 3. He will also oversee my self studies in machine learning, functional data analysis, and TDA. Dr. Cheng will continue to lend expertise in histopathology and provide equipment in support of the micro-CT experiments detailed by Aim 1. Their strong relationship as mentors and collaborators has already contributed to a publication and will continue to benefit from our monthly meetings. It is their understanding and flexibility that allowed me to perform imaging experiments that parted from the lab's status quo and generated the preliminary data for this grant proposal. Under this flexible collaboration I will continue to lead imaging experiments, and I will personally conduct additional phase-contrast imaging research at synchrotron beamlines. They have provided me with research opportunities I am grateful for - due to their collaboration my research takes me from the bench in Hershey to a synchrotron beamline in California and back to a whiteboard in Happy Valley.

In addition to Dr. Silverman and Dr. Cheng, I am supported by a strong thesis committee, faculty group, and broader research team who will support, collaborate with, and train me throughout the duration of this award. Several of these faculty maintain an open-door policy and have helped me to design experiments and prepare samples (Dr. Warrick, Dr. Gunther, and Dr. Hu). Dr. Warrick is the Chief of Anatomic Pathology at the College of Medicine (see letter of reference and biosketch), and he has directly overseen our sample selection and preparation in the proposed work. Dr. La Riviere (see letter of reference and biosketch) is a long-time collaborator of Dr. Cheng and Dr. Silverman, and we have already initiated multiple projects focused on the improvement of phase-contrast imaging and its applications to the investigation of human cancer. I am thankful to have assembled a supportive, knowledgeable, and versatile team that will drive me to not only complete the proposed work, but also to meet and exceed my long-term career goal of becoming an independent physician-scientist.

\bibliographystyle{IEEEtran}
\bibliography{refs.bib} %Entries are in the refs.bib file
\end{document}
