\documentclass{NIHGrant}
\begin{document}
\part*{Selection of Sponsor and Institution}
% one page limit
I elected to pursue a career as a physician scientist after successfully defending my undergraduate honors thesis and performing full time research as a member of the Tolbert lab. I applied broadly to programs with strong computational research opportunities. I interviewed at Penn State College of Medicine (PSCOM) and discovered that this institution not only offered a wealth of opportunities in computational research, but also provided a holistic, supportive, and thorough training environment that extended to clinical medicine as well. The humanities curriculum at PSCOM is unmatched and interfaces with every aspect of our training - research and clinical initiatives are each grounded in problems that plague our patients.

I started my MD/PhD journey with a pre-matriculation rotation in the Cheng Lab that helped set in motion the early formation of this interdisciplinary project. Dr. Cheng is an outstanding research mentor with an extroardinary track record as a leader in his field. He has a long history not only of high-impact publications and NIH funding, but also of innovation in imaging and computation. The micro-CT experiments pioneered in his lab prior to and during my rotation were compelling, but the aspect of the Cheng lab that attracted me the most was the interdisciplinary team that he lead in his initiative for computational phenomics. For example, the lab has brought in experts in the field of medical physics such as Dr. La Riviere (see letter of reference and biosketch) for experimental design and guidance. Members of his lab conduct research in model organisms and human disease alike, exposing students to a wide array of expertise. Also, the lab recently pioneered the use of novel micro-CT detectors, generating 3D datasets that were entirely unique in their resolution to field-of-view ratio. \uline{I saw an opportunity within this lab to conduct research on the 3D imaging of human cancer biopsies. The unique technology and collaborative research environment cemented my decision to work with Dr. Cheng.}

I completed my next laboratory rotation by spending a month working with Dr. Silverman in the summer between the first and second year of medical school. I had first learned of Dr. Silverman's work through his collaboration and ongoing consultation with the Cheng lab, and I decided to pursue a rotation in his lab because of my background in computational work and the unique data analysis problems I had encountered during my first rotation, such as 3D segmentation within high-resolution images. Dr. Silverman immediately held me to a high standard while simultaneously investing time in walking me through detailed and difficult concepts in math and statistics. \uline{To this day he does not hesitate to work through problems on the whiteboard with myself or any of his students, and it is this investment in trainees that has guided me as a grad student and motivated me to pursue a career leading a team of my own.}

I have built a strong foundation ever since these rotations that will allow the co-sponsorship between Dr. Silverman and Dr. Cheng to be successful. We have a common goal in our work to conduct research that will help patients, pathologists, and the broader imaging community, and this guides our choices in the lab. Both Dr. Cheng and Dr. Silverman have encouraged my independence and curiousity while holding me accountable for and helping me understand my missteps. It is their understanding and flexibility that allowed me to perform imaging experiments that parted from the lab's status quo and generated the preliminary data for this grant proposal. They have provided me with research opportunities I will forever be grateful for - due their collaboration \textbf{my research takes me from the bench in Hershey to a synchrotron beamline in California and back to a whiteboard in Happy Valley}. Their ability as mentors to both look out for my success and push me out of my comfort zone towards rigor and consistent progress will contribute greatly to my growth into an independent physician-scientist.

In addition to Dr. Silverman and Dr. Cheng, I am supported by a strong thesis committee, faculty group, and broader research team who will support, collaborate with, and train me throughout the duration of this award. Several of these faculty maintain an open-door policy and have helped me to design experiments and prepare samples (Dr. Warrick, Dr. Gunther, and Dr. Hu). Dr. Warrick is the Chief of Anatomic Pathology at the College of Medicine (see letter of reference and biosketch), and he has directly overseen our sample selection and preparation in the proposed work. Dr. La Riviere (see letter of reference and biosketch) is a long-time collaborator of Dr. Cheng and Dr. Silverman, and we have already initiated multiple projects focused on the improvement of phase-contrast imaging and its applications to the investigation of human cancer. I am thankful to have assembled a supportive, knowledgeable, and versatile team that will drive me to not only complete the proposed work, but also to meet and exceed my long-term career goal of becoming an independent physician-scientist.

\bibliographystyle{IEEEtran}
\bibliography{refs.bib} %Entries are in the refs.bib file
\end{document}
