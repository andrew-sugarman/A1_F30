\documentclass{NIHGrant}
\begin{document}

\part*{Respective Contributions}
% elaborate on post-comprehensive discussions  with Dr. Warrick?
% is it reasonable to say this proposed work is modeled after silverman R01?
The proposed research strategy is modeled after the questions investigated by the Silverman Lab. The group investigates the measurement of complex biomedical data (tools for sequence count data analysis, applied Bayesian statistics, partially identified models), working to develop statistical methods for reliable inference in the case of high-dimensional, unconventional data. Dr. Silverman is the primary sponsor of this project as we set out to improve the understanding of a fundamental problem in cancer biology and treatment by leveraging unique yet rigorous methods development with broad implications.
% the silverman lab investigates methods of measuring complex biomedical data, and subsequently develops creative solutions to enhance their rigor

The goals and expected outcomes of the proposed research also build on studies published in the Cheng Lab, specifically the work by prior MD/PhD student Yifu Ding (Ding et. al 2019) and PhD student Maksim Yakovlev (Yakovlev et. al 2023, 2024). However, mico-CT in general has only been applied to soft tissue cancer research relatively recently. In developing this proposal, I conducted a literature review and observed two critical gaps in knowledge that our team was uniquely positioned to contribute to: identifying repeatable phase-contrast imaging parameters capable of generating 3D images of whole-biopsies at diagnostic resolution and a need for rigorous, interpretable means of quantifying the phenotypes within these datasets. These two research questions are directly relevant to my long term goal of contributing to cancer research, and I identified micro-CT as a technology with increasing capability to improve our ability to measure and potentially diagnose malignancy. \emph{This proposal builds on current literature in micro-CT and combines the technology with novel statistical methods to advance our understanding prostate cancer and other malignancies.} This work does so while \emph{prioritizing my training in statistics and programming.}

The preliminary data collected in collaboration with Dr. Warrick (see letter of reference and biosketch) came from successful phase-contrast imaging experiments presented in my comprehensive exam, which I successfully defended in August of 2023. These data support the feasibility of this proposal and I will continue to build on this foundation with Aims 1 and 2. Dr. Warrick, Dr. Edward Gunther (see letter of support), and Dr. Jiafen Hu (see letter of support) will contribute additional samples as we investigate additional solid tumor phenotypes with PBCT. I will lead experiments at Lawrence Berkeley National Lab and Argonne National Lab to complete Aim 1, along with data acquisition, visualization, and processing for aim 2. Dr. Silverman and Dr. Warrick will mentor and assist me through the study design and statistical analysis of aim 2, and Dr. Silverman will directly oversee algorithm development and programming in aim 3. I will review results and prepare manuscripts in close discussion with Dr. Silverman, Dr. Cheng, Dr. Warrick, and Dr. La Riviere. I am thankful to be supported by a multi-disciplinary research team that is led by the close guidance of Dr. Silverman, who has set a great example from the start of how I aim to run my own future lab as an independent investigator. The experience I will gain from the proposed work will be elementary to my goal of leading interdisciplinary collaborations at the cutting edge of imaging and cancer research.

\bibliographystyle{nihunsrt}
\bibliography{refs.bib} %Entries are in the refs.bib file
\end{document}



%%% ask Caryn again about where to include IRB study information if at All
%%% mention prelim data in Dr. warrick letter
%%% fix RO1 number?
