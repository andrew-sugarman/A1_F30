\documentclass{NIHGrant}
\begin{document}
%%% ADD SECTION ABOUT DR. SILVERMAN's LABORATORY
%%% ADD SECTION SPECIFICALLY ABOUT MAXS DETECTOR

\part*{Facilities and other Resources}
\emph{The overall research and scientific environment available to me will strongly support the proposed research through both technological resources and abundant collaboration.}
\subsubsection*{Laboratory:} I primarily work in Dr. Keith Cheng's laboratory (co-sponsor) located on the 7th floor of the Biomedical Research Building in Hershey Medical Center. Dr. Cheng has provided me with a personal desk and access to multiple printers and whiteboards, in addition to a shared workstation equipped with 64Gb of RAM and Quadro RTX5000 GPU dedicated for image processing and reconstruction.

The Silverman lab has also provided me with a dedicated workspace that includes a desk, monitor, keyboard, and mouse that I utilize when I commute to University Park weekly. Our lab is located within the Westgate building at the
Pennsylvania State University, and is a short walk down the hallway from Dr. Silverman's office where we regularly meet in person and at the whiteboard to discuss our research.  Dr. Silverman has also provided me with my primary work computer, a Lenovo Thinkpad X1 Carbon with an intel i7 processor, 32GB of ram, and 1 terabyte of storage. Apart from my personal computing resources the Silverman lab also maintains access to the Roar Computing Cluster at University Park.

\subsubsection*{Computing Resources:} There three image reconstruction computers used by the Cheng Lab: i) (Recon 1) running Windows 11 Enterprise,  with 256Gb RAM, 40TB storage and GTX1080i GPU; ii) (Recon 2) with dual boot for Windows 11 Enterprise and Ubuntu, 256Gb RAM, 48TB and nVIDIA RTX8000; iii) (Recon 3) with Windows 11 Enterprise, 256Gb RAM, 48Tb and two nVIDIA A6000 GPU with nvlink. n. Connectivity between the above workstations, the scanner computer and network storage is gigabit Ethernet. The lab is also equipped with one Wacom Cintiq Pro 24, 6 Wacom Intous Pro and 1 Wacom One for segmentation. The laboratory also has 4 iMacs, and one dedicated workstation for genome analysis with 64GB of RAM and 12 TB of storage. The lab is equipped with 2 local network storage units with 80Tb and 40TB respectively. There is also one Canon color laser printer and scanner and an HP all-in-one color printer and scanner. The lab also has an Aperio AT2 slide scanner with a dedicated computer.

Penn State College of Medicine hosts a High Performance Computing (HPC) system that was purchased to provide researchers the computational and storage tools needed to efficiently and effectively process data. The HPC system is dedicated for use by College of Medicine researchers for processing genomic, DNA sequencing, imaging, and other scientific analysis. To comply with requirements for managing grant-supported research data, we will utilize the HPC system for computational and storage services. The Penn State College of Medicine’s Research Informatics department provides full support of both the operation and maintenance of the HPC environment. The HPC system is physically located on the Penn State Hershey campus and the data center was designed to Tier III data center standards. The system contains 3 administrative, 10 standards, and 3 high memory compute nodes. The system provides 1 Petabyte of enterprise storage that is divided into 100 Terabytes (TBs) of high-speed scratch space and 900 TBs of usable storage space. The ten compute nodes provide 240 2.5GHz Intel v3 cores (480 threads) with a total of 2560 Gigabytes (GBs) of RAM and the three high memory nodes provide 2.3 GHz 96 v3 cores (192 threads) with a total of 2304 GBs of RAM. The total compute capacity and total RAM of the system is 4864 GBs with the standard and high memory compute nodes providing 10.5 GBs of RAM and 24 GBs of RAM per core respectively. For additional computing tasks, I have access to the Roar computing system at University Park. I specifically am able to use the Silverman Lab's Roar cluster allocation through the Institute for Computational and Data Science at the College of IST. I will have access to the entire cluster in addition to priority (<1min wait time) access to my personal allocation which includes one high-memory node (40 cores, 1TB RAM),
two standard-memory nodes (each with 40 cores and 256GB RAM), 10TB Active Group Storage, and 20TB
Nearline/Archive Storage.


\subsubsection*{Institutional Environment:} Pennsylvania State University (Penn State) is the land grant institution of Pennsylvania. This proposal includes work conducted at both the College of Medicine campus and the University Park campus, the latter of which is the largest of Penn State's 24 campuses. Of note, Penn State University Libraries rank among the
top 10 North American research libraries based on the Association of Research Libraries Library Investment
Index Rankings. The library system consists of 36 libraries at 24 locations throughout the Commonwealth of
Pennsylvania. The University Libraries house a collection of nearly 6 million items, with annual additions of roughly
100,000 volumes. The libraries have access to 579 online databases and other e-resources and subscribe to
nearly 118,000 online, full-text journals.

Penn State College of Medicine, where I will primarily work for the duration of this proposal, also provides an elite setting for me to grow as a physician scientist. I both live and primarily work within a five minute walk each from the Emergency Department where I conduct my clinical exposure training, the Penn State Cancer Institute where I regularly shadow, and the Department of Pathology where I conduct research and read slides with my mentors. Because of Penn State's well-known collaborative culture across both campuses, I will interface with interdisciplinary graduate students, faculty, post doctoral researchers, physicians, and other healthcare providers that will enhance my training as a physician-scientist. Penn State will allow me to not only develop expertise at the bench and within the field of oncology, but will also encourage me to become well versed across disciplines.

\bibliographystyle{IEEEtran}
\bibliography{refs.bib} %Entries are in the refs.bib file
\end{document}
\newpage
\part*{Equipment}


\subsubsection*{PSCOM MD/PhD Program:}

%%% General Reference Link %%%
https://grants.nih.gov/grants/how-to-apply-application-guide/forms-h/fellowship-forms-h.pdf

%%% Rec computer specs %%%
1. AMD Ryzen Threadripper 3960X 24-Core Processor
2. 160GB ram
3. NVIDIA RTX A6000 graphics card with 48gb dedicated VRAM
4. > 80 terabytes of dedicated storage

%%% Rec 1 specs %%%
1. AMD Ryzen Threadripper 3970X 32-Core Processor
2. 256GB ram
3. NVIDIA RTX A6000 graphics card with 48gb dedicated VRAM
4. >100 terabytes of dedicated storage

%%% Rec 3 specs %%%
1.
