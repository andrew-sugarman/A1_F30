\documentclass{NIHGrant}
\begin{document}

\part*{Equipment}
\subsubsection*{Micro-CT Imaging Equipment:} \uline{The Cheng lab maintains a laboratory-based micro-CT imaging system with a 10K x 7K CMOS detector with specially-designed optics:} Dr. Cheng led the design, building, integration and testing of a custom microCT system with image quality, resolution and throughput sufficient for tissue phenotyping of mm- to cm-scale specimens. The system is enclosed by a custom lead and steel enclosure for the standard laboratory space. To ensure precision of motion we custom-designed a matched rotational and linear actuator system for sub-pixel, helical, and mosaic helical scans of large specimens of tens of mm in size. Because of our unique need for large field-of-view, high-resolution, and high-throughput, we designed a custom optical system which has a 5 x 3.5 mm$^{2}$ field of view (FOV), a 0.5$\mu$m isotropic voxel size and a process to fabricate scintillator wafers with 3-fold higher sensitivity improvement. In addition to the 5mm system, the Cheng lab has also developed an additional detector system with a 10mm FOV and 0.7$\mu$m isotropic voxel size.

The Cheng lab also provides secure luggage including 5 padded Pelican cases to bring this custom hardware to LBNL for synchrotron imaging experiments. The lab also ships a reconstruction computer with Windows 11 Enterprise, 256Gb RAM, 48Tb and a nVIDIA A6000 GPU. Additional equipment such as portable storage and tools for assembly and sample mounting are also provided by the Cheng Lab.

\subsubsection*{Research Materials:} The Silverman lab provided me with a dedicated workspace that includes a desk, monitor, keyboard, and mouse to support productivity during my time at University Park. Dr. Silverman has also provided me with my primary work computer, a Lenovo Thinkpad X1 Carbon with an intel i7 processor, 32GB of ram, and 1 terabyte of storage. The lab will also provide me with all necessary textbook materials to complete the guided learning and self-study portions of this proposal.

\bibliographystyle{IEEEtran}
\bibliography{sugarman_f30.bib} %Entries are in the refs.bib file
\end{document}
