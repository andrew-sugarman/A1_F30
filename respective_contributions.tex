\documentclass{NIHGrant}
\begin{document}

\part*{Respective Contributions}
%\textbf{add detail}
The proposed research training plan was modeled after foundational questions intersecting Dr. Warrick's clinical practice, Dr. La Riviere's knowledge of medical imaging, Dr. Cheng's expertise in micro-CT, and Dr. Silverman's development of statistical methods for complex biomedical data. Driven by my research and clinical interests in cancer, I spearheaded the assembly of this interdisciplinary team to supplement Dr. Cheng and Dr. Silverman's pre-existing collaboration. After my pre-matriculation rotation in the Cheng lab developing an automated approach to blood cell segmentation in 3D, I sought out Dr. Silverman's expertise in statistics to deepen my understanding of machine learning. Dr. Silverman's lab was an ideal training environment for me given my goals as an aspiring investigator in translational applications of computational biology. I have extensively studied probability theory, uncertainty calculation, and applied topological data analysis (TDA) in the Silverman lab. Dr. Silverman has also provided training in scientific writing and presentation skills through feedback throughout this application and several talks I have delivered in lab meeting and my comprehensive exam. Dr. Cheng has also provided invaluable opportunities in support of this award, and \emph{I have conducted 5 synchrotron trips to perform micro-CT experiments on behalf of the Cheng lab, leading 2 of them with a focus on propagation-based phase-contrast approaches.} In collecting the data presented in this proposal, I have already acquired skills across disciplines including synchrotron imaging, sample preparation and histology, and data science. The combination of these research training experiences have enabled me to contribute to a publication in \textit{eLife}.

Dr. Warrick's expertise has been instrumental to the selection and analysis of prostate cancer samples, and he has provided invaluable training in the basic science underlying prostate cancer disease progression and diagnosis. \emph{I regularly conduct joint meetings with Dr. Silverman and Dr. Cheng, and we routinely consult with Dr. Warrick as a group to ensure our research is fundamentally driven by his clinical practice.} Preliminary data supports the feasibility of this proposal and I will continue to build on this foundation with Aims 1 and 2.  Dr. Warrick has selected prostate needle core biopsy samples in accordance with our IRB protocol and provided them to me for preparation and scanning. He, along with Dr. Edward Gunther (see letter of support), and Dr. Jiafen Hu (see letter of support) will contribute additional samples as we continue investigate a variety solid tumor phenotypes with PBCT in support of Aim 1. \emph{I will lead additional experiments at Lawrence Berkeley National Lab and Argonne National Lab to complete Aim 1, along with data acquisition, visualization, and processing for Aim 2}. Dr. Cheng will provide equipment such as the 0.5$mu$m resolution, 5mm field-of-view detector system, training in sample prep, and training in micro-CT image analysis in support of Aim 1. Dr. Silverman and Dr. Warrick will mentor and assist me through the study design and statistical analysis of Aim 2, and Dr. Silverman will provide instruction in algorithm and method development in support of Aim 3. Dr. Silverman will continue an active role in sharpening my skills in scientific communication by providing iterative individual feedback over lab meeting, departmental, and conference presentations. I will review results and prepare manuscripts in close discussion with Dr. Silverman, Dr. Cheng, Dr. Warrick, and Dr. La Riviere. I am thankful to lead this collaboratorive project with the close guidance of Dr. Silverman, who has set a great example of how I aim to run my own lab as an independent investigator. The proposed research training plan will be elementary to my goal of leading interdisciplinary collaborations at the cutting edge of imaging and computation in cancer research.

\bibliographystyle{nihunsrt}
\bibliography{refs.bib} %Entries are in the refs.bib file
\end{document}
1. Highlight that I catalyzed this collaboration dirven by your interests and supported by your advisors
2. The three of us regularly meet (Dr. Silverman + keith regularly meet )
3. this is a multidisciplinary project and I have already acquired skills across disciplines (synch imaging, data science work elife)
         4. what have been my contributions - how have the advisors supported that, and how they will support
         5. dr. cheng imaging resources technique facilities
         6. dr. warrick framed initial research question and provided samples
         7. dr silverman - there are goals I have set that led me to seek out Dr. silverman
         8. the ultimate project is an interdisciplinary collaboration highlighting the expertise of my mentorship team, driven by my own interests in technically demanding cancer research. Dr. so and so's contributions are

Describe the collaborative process between you and your sponsor/co-sponsor(s) in the development, review, and editing of this Research Training Plan. Also discuss your respective roles in accomplishing the proposed research.

 Through my thesis committee meetings and comprehensive exam, I learned from Dr. Hohl and Dr. Warrick who each participate in translational cancer research.
- I pursued this collaboration after laboratory rotations in the Cheng and Silverman labs. Dr. Cheng and Dr. Silverman had established an early stage collaboration at the time of my rotation in the Cheng lab, wherein I was performing micro-CT image segmentation for a project later published in eLife.
